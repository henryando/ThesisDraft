Large-scale quantum networks will require reliable single-photon sources for distributing entanglement over long distances. The \erbium ion is a good candidate for this purpose as it has an optically accessible transition at 1550 nm, the so-called ``telecom'' wavelength at which light propagates through optical fibers with minimal loss. However, this transition is electric-dipole forbidden and thus spontaneous emission from it is slow (around 100 Hz).
Such a bandwidth constraint would be crippling in a real photonic quantum system. The Thompson lab has bulk-doped \erbium ions into \YSO and then coupled them to silicon nanophotonic cavities, enabling the observation of single photon emissions and enhancing the spontaneous emission rate by a factor of 600 through the Purcell effect \cite{Dibos2017}.
However, the bulk-doped \ErYSO has two downsides. One is that \YSO has strong nuclear spins which limit the  $T_{1}$ and $T_{2}$ spin coherence times of the \erbium ions. The other is that the lower bound on the density of \erbium possible through the bulk-doping process is still too high for individual \erbium ions to be addressed optically.
These issues have motivated a search for new host crystals which can be doped with \erbium through ion implantation, hopefully resulting in substantially lower concentrations of \erbium ions than would be possible through bulk doping. In this work, we characterize the spectral properties of \erbium implanted in various new host crystals, and discuss new techniques for improving the resolution of future such measurements.